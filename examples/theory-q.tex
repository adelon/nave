\begin{theory}
   An arithmetic consists of
   a set $N$, an element $\zero \in N$,
   $S : N\to N$,
   $+: N\to N$
   and $\cdot : N\to N$.
\end{theory}

\begin{theory}
   A Robinson arithmetic is an arithmetic $(N,\zero,S,+,\cdot)$
   such that
   \begin{enumerate}
      \item $\forall x.\, S(x)\neq \zero$
      \item $\forall x y.\, S(x) = S(y) \to x = y$
      \item $\forall y.\, y = \zero \lor \exists x.\, S(x) = y$
      \item $\forall x.\, x + \zero = x$
      \item $\forall x y.\, x + S(y) = S(x+y)$
      \item $\forall x.\, x\cdot \zero = \zero$
      \item $\forall x y.\, x\cdot S(y) = (x\cdot y) + x$
   \end{enumerate}
\end{theory}

Throughout, let $N$ be a Robinson arithmetic.

\begin{theorem}
    $S(S(\zero))\cdot S(S(\zero)) = S(S(S(S(\zero))))$.
\end{theorem}