% The mutilated checkerboard problem


% Combine as simple finite inductive type:
%
% Induces the notions 'a rank / ranks' and 'the (set of) ranks'.
% Also induces appropriate axioms in the FOL interpretation.
%
\begin{inductive}
   A rank is one of
   $\rankOne$, $\rankTwo$, $\rankThree$, $\rankFour$, $\rankFive$, $\rankSix$, $\rankSeven$, or $\rankEight$.
\end{inductive}

Let $r, r_1, r_2$ denote ranks.


% Analogous to 'rank':
%
\begin{inductive}
   A file is one of
   $\fileA$, $\fileB$, $\fileC$, $\fileD$, $\fileE$, $\fileF$, $\fileG$, or $\fileH$.
\end{inductive}

Let $f, f_1, f_2$ denote files.


% TODO Unsure what would be a good formulation.
% It would be nice to be able to define this directly and have
% translator generate the appropriate axiom.

%\begin{signature}
%   A square is a notion.
%\end{signature}
%\begin{axiom}
%   $(f,r)$ is a square.
%\end{axiom}

%\begin{definition}
   %A square is a pair of a file and a rank.
   %Squares are pairs of files and ranks.
%\end{definition}

\begin{definition}
   The checkerboard is the set of all squares.
\end{definition}

Let $s, s_1, s_2, s_3, s_4$ denote squares.




% Needs ForTheL-style notions with variables!
\begin{definition}
   The mutilated checkerboard is the set of all squares $s$
   such that $s \neq (\fileA, \rankOne), (\fileH, \rankEight)$.
\end{definition}




% TODO or define the predicates inductively?

\begin{signature}
   $r_1$ can be vertically adjacent to $r_2$.
\end{signature}

\begin{axiom}
   If $r_1$ is vertically adjacent to $r_2$ then $r_2$ is vertically adjacent to $r_1$.
\end{axiom}


% This is using the Naproche convention of only pairing up items in sequence.
% To pair up all items (which we do not want to do) one would need to use the word 'pairwise'.
\begin{axiom}
   $\rankOne, \rankTwo, \rankThree, \rankFour, \rankFive, \rankSix, \rankSeven, \rankEight$
   are vertically adjacent to each other.
\end{axiom}


\begin{signature}
   $f_1$ can be horizontally adjacent to $f_2$.
\end{signature}

\begin{axiom}
   If $f_1$ is horizontally adjacent to $f_2$ then $f_2$ is horizontally adjacent to $f_1$.
\end{axiom}


\begin{axiom}
   $\fileA, \fileB, \fileC, \fileD, \fileE, \fileF, \fileG, \fileH$
   are horizontally adjacent to each other.
\end{axiom}


\begin{definition}
   A square $s_1$ is adjacent to $s_2$ iff
   there exist $f_1, r_1, f_2, r_2$ such that
   $s_1 = (f_1, r_1)$, $s_2 = (f_2, r_2)$ and on of the following holds:
   $f_1 = f_2$ and $r_1$ is vertically adjacent to $r_2$;
   or $r_1 = r_2$ and $f_1$ is vertically adjacent to $f_2$.
\end{definition}



\begin{definition}
   A domino is a set $D$
   such that $D = \{ s_1, s_2 \}$
   for some adjacent squares $s_1, s_2$.
\end{definition}


\begin{definition}
   A tiling is a pairwise disjoint set of dominos.
\end{definition}

\begin{definition}
   Let $A$ be a subset of the checkerboard.
   A tiling of $A$ is a tiling $T$
   such that every square $s$
   is an element of $A$ iff
   $s$ is an element of some element of $T$.
\end{definition}


\begin{signature}
   A square can be black.
   A square can be white.
\end{signature}

% Using '_ xor _' instead of '_ iff not _'
\begin{axiom}
   A square $s$ is either black or white.
\end{axiom}

\begin{axiom}
   If $s_1$ is adjacent to $s_2$
   then $s_1$ is black iff $s_2$ is white.
\end{axiom}

\begin{axiom}
   $(\fileA, rankOne)$ is black.
   $(\fileH, rankEight)$ is black.
\end{axiom}


\begin{signature}
   Let $s$ be a square.
   Then $\swap(s)$ is a square.
\end{signature}
